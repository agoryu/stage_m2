\subsection{Segmentation}
%recuperation d'un objet dans l'environnemnt -> interaction utilisateur
%dire que la création d'une base de connaissance est plus utile que pour le corps humain
La première étape lors de notre projet va être de segmenter les images que nous recevons de 
notre caméra. Les informations contenues dans une image 3D sont nombreuses et nous avons
besoin de déterminer les éléments important pour nos traitement et le reste. Dans notre scène,
nous avons besoin des objets proches ou du corps de la personne en face de la caméra, mais 
l'environnement autour des ces objets clés n'est pas important et doit être supprimer pour
gagner du temps lors de nos traitements en supprimant de l'information à traiter. Certaines
techniques de segmentation sont utilisable aussi bien pour les objets que pour le corps humain.
Cependant, il existe quelques différences entre le corps humain et un objet comme le fait que le
corps humain soit flexible et qu'il bouge alors qu'un objet est rigide et il est le plus souvent 
statique dans les données que nous récupérons pour ce projet.
 
\subsubsection{Corps humain}

\subsubsection{Scène intérieur}

\subsection{Reconnaissance d'objets}
%FPFH
%SHOT
\subsubsection{Descripteur}
\subsubsection{SVM} %TODO mettre le nom complet
\subsubsection{Bag of word}

\subsection{Positionnement de modèle}
%moment d'inertie pour la position des membres
%PCA
