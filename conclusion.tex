Durant ce stage, j'ai effectué deux projets permettant de réaliser la segmentation d'objets complexes comme le corps
humain ainsi qu'une scène intérieure à partir d'image 3D provenant d'une caméra Kinect. L'objectif était de créer
des interfaces permettant de faciliter la tâche des professions artistiques en leur permettant d'utiliser 
des objets du monde réel dans un environement virtuel. Dans la première application sur le corps humain,
nous avons cherché à récupérer les informations du corps de la personne en face de la caméra et segmenter
ces informations. Les informations ainsi segmentées pouvaient ensuite être remplacées par l'utilisateur afin de 
pouvoir mettre des éléments virtuels sur un corps venant du monde réel. Pour réaliser ce projet, nous avons 
utilisé les fonctionnalités fournis dans le SDK de la Kinect pour filtrer les données. Au vu de la différence
des données entre le nuage de points que nous récupérons avec notre caméra et les modèles 3D que nous utilisons,
nous avons privilégié une méthode basée sur l'utilisation des caractèristiques de la boîte englobante pour
déterminer la position et l'orientation du nuage de points de la partie du corps à remplacer. Ces informations
nous permettent ainsi de transformer le modèle 3D pour que celui-ci soit exactement à la même position que
le premier nuage de points. Les limites de notre méthode sont qu'une boîte englobante est une représentation 
beaucoup trop générale pour une forme, et donc l'orientation du nouveau modèle possède souvent une erreur de 90\degre
sur l'un des trois axes de la boîte. Pour l'application finale, nous avons pensé proposer à l'utilisateur de 
corriger ce problème en sélectionnant l'axe qui fait défaut ainsi l'application se chargerait d'appliquer la
bonne rotation au modèle.\\

La seconde applications avait pour but de prendre un environnement virtuel dans lequel nous pouvons rajouter des éléments 
du monde réel. Ces objets du monde réel viennent de l'acquisition d'une pièce via la caméra Kinect. Là encore, le processus nécessitait 
la segmentation des données afin de détecter les objets présents dans la pièce réelle. La segmentation est réalisée 
par l'utilisateur qui doit sélectionner l'objet qu'il veut importer dans le monde virtuel. Pour que l'application
reconnaisse l'objet qui a été sélectionné, nous avons utilisé un algorithme d'apprentissage automatique, les SVM,
combiné avec l'approche du \textit{bag of words} formé à partir du descripteur FPFH.\\
%TODO ajout des résultat de la méthode.

Lors de ce stage, j'ai eu l'occasion de perfectionner les connaissances que j'avais acquises lors de mon master.
Les notions que j'avais vues de reconnaissance d'objet m'ont évidemment était très utile durant le stage, mais 
beaucoup de concepts n'ont pas été approfondis durant les cours par manque de temps. J'ai pu perfectionner mes 
connaissances concernant les algorithmes d'apprentissage automatique et j'ai pu voir le fonctionnement de plusieurs
descripteurs.\\ 

Le plus compliqué lors de ce stage a été de comprendre et tester des méthodes proposées dans des publications scientifiques.
Certaines méthodes n'étaient pas très complexes et faciles à tester, mais de nombreuses publications se reposaient sur des connaissances
mathématiques qui n'ont pas été vues durant mon cursus scolaire. Cependant, cette expérience m'a appris à réaliser un état de l'art des 
solutions existantes sur un sujet très spécifique et à tester les méthodes proposées dans la littérature. Ma seconde difficulté,
lors de ce stage, est que le sujet était assez vaste et qu'il ne fallait pas se perdre dans des méthodes qui s'éloignaient du sujet.
Cela m'a appris à rester concentré sur l'objectif principal en prévoyant et en organisant les étapes du projet.\\ 

Ce stage m'a fait redécouvrir l'environnement de la recherche dans lequel je souhaitais m'orienter. Bien que j'ai réussi à surmonter
les difficultés que j'ai rencontrées durant ce stage, il m'a permis de constater que l'environnement ne me convenait pas. La partie 
\og état de l'art \fg, bien que très intéressante, a été assez complexe à réaliser. De plus, j'ai ressenti une certaine frustration
concernant les résultats du projet. J'ai passé énormément de temps à tester des méthodes difficiles à implémenter et je n'obtenais
que rarement des résultats intéressants pour mon projet. Ce stage m'a permis de comprendre que le monde de l'entreprise me
convenait mieux, grâce à son organisation et à sa structure qui permet de fournir des résultats 
rapides. Je pense qu'à l'heure actuelle, je ne suis pas encore capable de structurer un projet pour
le mener à bien avec autant d'autonomie. Même si la recherche est un travail d'équipe, les projets 
s'organisent souvent seul, et c'est sur ce point qu'il me reste encore à progresser.
