Durant ce stage j'ai réalisé deux projets permettant de réaliser la segmentation d'objet complexe comme le corps
humain et d'une scène intérieur à partir d'image 3D provenant d'une caméra Kinect. L'objectif était de créer
des interfaces permettant de faciliter la tâche des professions artistiques en leur permettant d'utiliser 
des objets du monde réelle dans un environement virtuel. Dans la première application sur le corps humain,
nous avons cherché à récupérer les informations du corps de la personne en face de la caméra et de segmenter
ces informations. Les informations ainsi segmenté pouvait ensuite être remplacé par l'utilisateur afin de 
pouvoir mettre des éléments virtuel sur un corps venant du monde réel.\\

La seconde applications avait pour but de prendre une pièce virtuelle dans laquel nous pouvons rajouté des éléments 
du monde réel. Ces objets du monde réel vienne du scanne d'une pièce. La encore le processus nécessitait la segmentation 
des données afin de détecter les objets présents dans la pièce réel.

\subsection{Evaluation des solutions}
Les deux applications que j'ai développé n'ont pas aboutit au résultat attendu. Le premier projet n'a pas été terminé
par faute de temps. 


Le problème sur l'orientation du modèle 3D a pris beaucoup de temps et très peu de solution à ce
type de problème ont été envisagé dans la littérature. 
\subsection{Bénéfice de ce stage}
\subsection{Perspective}
