%\subsection{Récupération de l'image du corps humain}
%SDK de la kinect avec skelette
%\subsection{Préparation de l'image}
%suppression du bruit et diminution du nombre de point -> pcl
%\subsection{Segmentation du corps humain}
%distance geodesic avec dikjstra -> pb nombre de voisin
%superpixel
%\subsection{Reconnaissance des parties du corps}
%descripteur D2 avec model 3D
%FPFH
%\subsection{Appariement d'un model 3D}
%scale à partir d'une bounding box
%ICP
%moment d'inertie
%\subsection{Résultat des expérimentations}
%montrer plusieurs images de la segmentation et montrer ce qui ne convient pas
%dire que les méthode utilisé seront utilisé dans la suite du projet 
\subsection{Objectif}
%on cherche a ne pas utiliser la kinect
\subsection{Délimitation du corps humain}
%prétraitement
\subsection{Calcule de la distance géodésique}
\subsection{Calcule de descripteurs}
%d2
\subsection{SDK de la Kinect}
%récuperation du corps de la personne depuis une image kinect -> a partir des info de la kinect -> le mettre plutot dans les travaux réalisé
%labelisation en fonction de la distance du point avec le joint
\subsection{Positionnement}
%icp, fpfh
%dire qu'il n'est pas possible d'utiliser la distance geodesic pour le positionnement a cause du temps et qu'on ne peut calculer la distance geodesic
%sur le mesh de remplacement
