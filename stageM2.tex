\documentclass[a4paper,11pt]{article}
\usepackage[T1]{fontenc}
\usepackage[utf8]{inputenc}
\usepackage{lmodern}
\usepackage[francais]{babel}

\title{Approche sémantique de segmentation et de recherche interactive par le contenu issu d’une caméra de profondeur}
\author{Elliot Vanegue}

\begin{document}

\maketitle
\newpage
\tableofcontents
\newpage

\begin{abstract}
\end{abstract}

%%%%%%%%%%%%%%%%%%%%%%%%%%%%%%%%%%%%%%%%%%%%%%%%%%%%%%%%%%%%%%%%%%%%%%%%%%%%%%%%%%%%%%%%%%%%%%%%%%%%%%%%%%%%%%%%%%%%%%%%%
\section{Introduction}
\subsection{Contexte}
%objectif du stage
%labo de recherche
%labo 3D-sam
Durant notre master IVI\footnote{Le master Image Vision Interaction est 
une spécialité du master informatique de l'université de Lille 1}, 
nous avons l'occasion de réaliser un stage de fin d'étude. J'ai choisi de réaliser ce stage
dans le laboratoire 3D-SAM spécialisé dans l'acquisition et le traitement d'image 3D 
à partir de capteur 3D de type Microsoft Kinect. Leur principaux travaux porte sur
l'analyse de forme d'objet 3D et la modélisation des variation des formes dans des
vidéo 3D. 
%J'ai choisi de réaliser mon stage dans un laboratoir de recherche, car je souhaite voir 

\subsection{Sujet du stage}

\subsection{Problèmatique}

\subsection{Déroulement du stage}
%parler de pcl et opencv (opengl ?)


%%%%%%%%%%%%%%%%%%%%%%%%%%%%%%%%%%%%%%%%%%%%%%%%%%%%%%%%%%%%%%%%%%%%%%%%%%%%%%%%%%%%%%%%%%%%%%%%%%%%%%%%%%%%%%%%%%%%%%%%%%
\section{Etat de l'art}

\subsection{Segmentation d'un environnemnt 3D}
%récuperation du corps de la personne depuis une image kinect -> a partir des info de la kinect -> le mettre plutot dans les travaux réalisé
%recuperation d'un objet dans l'environnemnt -> interaction utilisateur

\subsection{Segmentation du corps humain}

\subsection{Reconnaissance des parties du corps humain}
%FPFH
%SHOT

\subsection{Appariement de partie du corps avec des models existant}
%moment d'inertie pour la position des membres
%PCA

\subsection{Reconnaissance d'objets}
%dire que la création d'une base de connaissance est plus utile que pour le corps humain

%%%%%%%%%%%%%%%%%%%%%%%%%%%%%%%%%%%%%%%%%%%%%%%%%%%%%%%%%%%%%%%%%%%%%%%%%%%%%%%%%%%%%%%%%%%%%%%%%%%%%%%%%%%%%%%%%%%%%%%%%%%
\section{Segmentation du corps}
\subsection{Récupération de l'image du corps humain}
%SDK de la kinect avec skelette
\subsection{Préparation de l'image}
%suppression du bruit et diminution du nombre de point -> pcl
\subsection{Segmentation du corps humain}
%distance geodesic avec dikjstra -> pb nombre de voisin
%superpixel
\subsection{Reconnaissance des parties du corps}
%descripteur D2 avec model 3D
%FPFH
\subsection{Appariement d'un model 3D}
%scale à partir d'une bounding box
%ICP
%moment d'inertie
\subsection{Résultat des expérimentations}
%montrer plusieurs images de la segmentation et montrer ce qui ne convient pas
%dire que les méthode utilisé seront utilisé dans la suite du projet 

%%%%%%%%%%%%%%%%%%%%%%%%%%%%%%%%%%%%%%%%%%%%%%%%%%%%%%%%%%%%%%%%%%%%%%%%%%%%%%%%%%%%%%%%%%%%%%%%%%%%%%%%%%%%%%%%%%%%%%%%%%%
\section{Reconstruction d'un environnement intérieur}
\subsection{Segmentation de l'environnement intérieur}
%segmentation par l'utilisateur
\subsection{Création d'une base de connaissance}
%utilisation opencv
%creation du dictionnaire
%calcul du descripteur des objets -> detail sur le nombre d'objet et sur le nombre d'exemplaire
%test sur une base composé de mesh et sur une autre composé de nuage de point
\subsection{Reconnaissance des objets} 
%simple comparaison avec le descripteur
%TODO voir ce qu'on peut mettre dans cette section

%%%%%%%%%%%%%%%%%%%%%%%%%%%%%%%%%%%%%%%%%%%%%%%%%%%%%%%%%%%%%%%%%%%%%%%%%%%%%%%%%%%%%%%%%%%%%%%%%%%%%%%%%%%%%%%%%%%%%%%%%%%
\section{Evaluation des méthodes}
%TODO aucune méthode d'évaluation déterminé

%%%%%%%%%%%%%%%%%%%%%%%%%%%%%%%%%%%%%%%%%%%%%%%%%%%%%%%%%%%%%%%%%%%%%%%%%%%%%%%%%%%%%%%%%%%%%%%%%%%%%%%%%%%%%%%%%%%%%%%%%%%
\section{Conclusion}
%resultat des deux projets
%ce que j'ai utilisé
%ce que j'ai appris
%comment me servir de mon stage pour après mes études

\end{document}
