\documentclass[a4paper,11pt]{article}
\usepackage[T1]{fontenc}
\usepackage[utf8]{inputenc}
\usepackage{lmodern}
\usepackage[francais]{babel}
\usepackage[top=2.5cm, bottom=2.5cm, left=2.5cm, right=2.5cm]{geometry}

%image
\usepackage{graphicx}

%url
\usepackage{hyperref}

\usepackage{array,multirow,makecell}
\setcellgapes{1pt}
\makegapedcells
\newcolumntype{C}[1]{>{\centering\arraybackslash }b{#1}}

\title{Approche sémantique de segmentation et de recherche interactive par le contenu issu d’une caméra de profondeur}
\author{Elliot Vanegue}

\begin{document}

\maketitle
\newpage

\section*{Remerciement}
\newpage

\begin{abstract}
Lors de ce stage, je vais réalisé deux applications dont le but est de modifier un environnement
virutelle grâce à des éléments du mon réel. Pour cela, je me sers de la caméra 3D Kinect v2
pour récupérer les informations du monde réel sous forme de nuage de points. Grâce à ces données
nous pouvons détecter des objets afin de les matérialiser dans un environnement virtuelle.
Ce projet vise à faciliter les outils de designer afin de leur permettre d'importer des objets
du monde réel dans leur projet pour éviter de remodéliser des objets existants. La première 
application récupère le corps de l'utilisateur dans le but de pouvoir en modifier certaine
partie. La seconde application permet de scanner un environnement intérieur pour ensuite 
sélectionné les objets que l'on souhaite ajouter dans un environnement 3D. Les deux applications
passe par une étape de segmentation, puis de détection et de reconnaissance d'objet. Les parties
les plus complexe à réaliser à l'heure actuel sont pris en charge par l'utilisateur au travers 
d'une interface que j'ai construit pour ces deux projets.\\

Pour ces projets j'ai utilisé les outils fournis dans le SDK de la Kinect, ainsi qu'une bibliothèque
pour le traitement de nuage de poing appelé PCL. Mes travaux se sont surtout concentré sur la reconnaissance
d'objet notamment dans la seconde application. Pour cette étape j'ai utilisé un descripteur appelé FPFH dans 
un sac de mot pour calculer un descripteur uniformisé et invariant en translation, rotation et changement
d'échelle. Afin de créer une base de connaissance j'ai utilisé un algorithme d'apprentissage automatique
appelé SVM.
\end{abstract}
\newpage

\tableofcontents
\newpage

%%%%%%%%%%%%%%%%%%%%%%%%%%%%%%%%%%%%%%%%%%%%%%%%%%%%%%%%%%%%%%%%%%%%%%%%%%%%%%%%%%%%%%%%%%%%%%%%%%%%%%%%%%%%%%%%%%%%%%%%%
\section{Introduction}
\subsection{Contexte}
%objectif du stage
%labo de recherche
%labo 3D-sam
Durant notre master IVI\footnote{Le master Image Vision Interaction est 
une spécialité du master informatique de l'université de Lille 1}, 
nous avons l'occasion de réaliser un stage de fin d'étude. J'ai choisi de réaliser ce stage
dans le laboratoire 3D-SAM spécialisé dans l'acquisition et le traitement d'image 3D 
à partir de capteur 3D de type Microsoft Kinect. Leur principaux travaux porte sur
l'analyse de forme d'objet 3D et la modélisation des variation des formes dans des
vidéo 3D. 
%J'ai choisi de réaliser mon stage dans un laboratoir de recherche, car je souhaite voir 

\subsection{Sujet du stage}

\subsection{Problèmatique}

\subsection{Déroulement du stage}
%parler de pcl et opencv (opengl ?)


%%%%%%%%%%%%%%%%%%%%%%%%%%%%%%%%%%%%%%%%%%%%%%%%%%%%%%%%%%%%%%%%%%%%%%%%%%%%%%%%%%%%%%%%%%%%%%%%%%%%%%%%%%%%%%%%%%%%%%%%%%
\section{Etat de l'art}
\subsection{Segmentation}
%recuperation d'un objet dans l'environnemnt -> interaction utilisateur
%dire que la création d'une base de connaissance est plus utile que pour le corps humain
La première étape lors de notre projet va être de segmenter les images que nous recevons de 
notre caméra. Les informations contenues dans une image 3D sont nombreuses et nous devons
déterminer les éléments important pour nos traitements. Dans notre scène,
nous avons besoin des objets proches ou du corps de la personne en face de la caméra, mais 
l'environnement autour des ces objets clés n'est pas important et doit être supprimer pour
gagner du temps lors de nos traitements en supprimant de l'information à traiter.
Une second segmentation est nécessaire pour le traitement du corps humain. Pour cette étape du projet,
nous devons segmenter le corps en plusieurs partie pour pouvoir par la suite les reconnaitres. Si cette
seconde segmentation n'est pas réalisé il ne nous sera pas possible de reconnaître les mains ou encore
la tête si nous ne savons délimité les parties du corps.  

\begin{figure}[!h]
  \begin{center}
    \includegraphics[width=8cm]{image/segmentation.png}
    \includegraphics[width=5cm]{image/bodySegmentation.png}
    \caption{Exemple de segmentation recherché pour une pièce intérieur et pour le corps humain}
  \end{center}
\end{figure}
 
\subsubsection{Scène intérieur}
De nombreux travaux ont été réalisé dans la segmentation d'image 2D avant que les caméras 3D ne soit
ouvert au grand public. Les premières méthode de segmentation reposaient sur la détection de contour
comme pour la méthode de P. Arbelaez et al\cite{2DSegmentation1}. Leur méthode repose sur le détecteur
de contour gPb qui est composé de d'un seuillage sur la luminance et sur la couleur et d'une détection
de texture. La fermeture des contours se fait ensuite en utilisant les superpixels. D'autres méthodes
2D utilise un simple seuillage en utilisant par exemple la méthode de N. Otsu\cite{Otsu} pour binariser
l'image et ainsi la segmenter.\\

%peut etre que l on peut rajouter des publi utilisant la depthmap
%voir si on sépare la depthmap et le nuage de point
Avec l'arrivé des caméras 3D de nombreuses recherche ont été effectué sur la segmentation d'image à partir
des information extraite de ce type de caméra. S.A.A Shah et al\cite{3DSegmentation1} utilisent les informations
de l'image de profondeur afin de calculer un vecteur sur chaque pixel. En applicant un seuillage sur la différence
des vecteurs ils obtiennent une segmentation de l'environnement qui leur permet de détecter des objets dans une pièce.
Il est possible à partir de l'image de profondeur de créer un nuage de point, ce qui permet d'obtenir les 
coordonnées 3D des points présents dans l'image de profondeur. Les informations qu'il est possible d'extraire
d'un nuage de point sont différentes et des méthodes de segmentation se sont développé autour de ces informations.\\

T. Rabbani et al\cite{pointCloudSegmentation} utilise les informations obtenus dans un nuage de point afin 
de calculer les normales de chaque point. Ils segmente ensuite l'image en comparant les normales et en appliquant
un seuillage sur cette comparaison. Si l'angle formé par les normales de deux points est super au seuil alors
les points appartiennent à deux régions différentes.\\

Nous pouvons voir que les méthodes cité précédemment sont efficace pour segmenter une scène comportant des objets,
mais elles ne sont pas applicable à un corps humain. Le principal défaut de ces méthodes pour le corps humain est 
que celui-ci est trop lisse. La différence entre les normales ou entre les vecteurs de pixel n'est pas assez important
et est trop instable pour que cela marche sur le corps humain.

\subsubsection{Corps humain}
La segmentation du corps humain est un sujet très complexe, car contrairement au objet celui-ci bouge et adopte
des postures différentes. La méthode la plus souvent utilisé pour résoudre cette problématique est de déterminer
la posture de l'utilisateur et lorsque cette posture est connu il est facile de déterminer les différentes partie
du corps. Ces méthodes nécessitent d'avoir une base de connaissance contenant de nombreuses postures qui doivent
être segmenter et labelisé avec les différentes parties du corps. J. Shotton et al\cite{kinectSegmentation} ont
d'abord créé une base d'apprentissage en calculant un descripteur\footnote{Voir \ref{descriptor}}
et une technique d'apprentissage automatique appelé forêt d'arbres décisionnels\cite{randomDecisionForest}. 
Lorsque l'utilisateur bouge, le descripteur utilisé précédemment est recalculé sur l'image courante et le résultat 
est comparé au posture de la base d'apprentissage. La posture ayant une valeur proche du résultat calculé précédemment 
est la posture de l'utilisateur. De nombreuses méthode utilise cette solution, mais elle modifie le descripteur
utilisé et la technique d'apprentissage automatique comme nous le verrons dans la suite de ce rapport.

%TODO apporter d'autre exemple + parler du descripteur de shotton

\subsection{Reconnaissance d'objets}
La reconnaissance d'objet est un sujet assez vaste dans le monde de l'imagerie et il existe de nombreuses
méthodes dans le domaine que ce soit pour des images 2D ou 3D. Dans le cas d'image 3D la méthode la plus utilisé
est le calcul de descripteur, ce qui correspond à un ensemble de caractèristiques représentant un objet spécifique.

\subsubsection{Descripteur}
\label{descriptor}
Le nombre de descripteur qui existe dans le domaine de l'image 3D est assez important c'est pourquoi pour ce rapport,
nous allons nous contenter de décrire seulement les plus utilisées. Le descripteur D2\cite{D2} est un des outils de 
comparaison de forme 3D les plus simple à réaliser. Il se repose sur le calcul de la distance euclidienne entre 
chaque point du modèle 3D. L'ensemble de ces distances permet de créer un histogramme 1D et de comparer ces histogramme
afin de reconnaître un objet. Ce descripteur fournit de bon résultat lorsque les objets à reconnaître sont très 
différents.\\

Le descripteur PFH\cite{PFH} (Point Feature Histograms) est un outil permettant de calculer la courbure moyenne d'un voisinage de point en utilisant
un histogramme multi-dimensionnel. Le calcul de la courbure et le fait que ce soit une généralisation permet d'être invariant 
en translation et en rotation, et permet également d'être moins sensible au bruit présent dans le nuage de point. Le voisinage 
dépend de la distance des points avec le point centrale et il ne peut exceder un certain nombre de voisin (voir Fig. \ref{fig:pfhNeighborhood}).

\begin{figure}[!h]
  \begin{center}
    \includegraphics[width=7cm]{image/PFH.png}
    \caption{Exemple de voisinage pris en compte dans le calcul de la courbure du descripteur PFH}
    \label{fig:pfhNeighborhood}
  \end{center}
\end{figure}

Le descripteur PFH calculé en un point correspond à la relation que ce point a avec l'ensemble des points de son voisinage. Cette relation
est la différence des normals entre deux points. Chaque classe de l'histogramme est composé de l'ensemble des points du voisinage dont 
la relation avec le point centrale est similaire. Une version amélioré du descripteur a été proposé par R. B. Rusu et al\cite{FPFH} appelé
FPFH (Fast Point Feature Histograms). Cette version est plus rapide, car elle calcul un descripteur PFH simplifié, puis elle construit
de nouveaux histogrammes à partir des histogramme simplifié précédent (voir Fig. \ref{fig:fpfhNeighborhood}).

\begin{figure}[!h]
  \begin{center}
    \includegraphics[width=7cm]{image/FPFH.png}
    \caption{Exemple de voisinage pris en compte dans le calcul de la courbure du descripteur FPFH}
    \label{fig:fpfhNeighborhood}
  \end{center}
\end{figure}

%TODO SHOT -> voir si on en parle
\subsubsection{Apprentissage automatique}
%svm
%random forest
\subsubsection{Bag of word}

\subsection{Positionnement de modèle}
%moment d'inertie pour la position des membres
%PCA


%%%%%%%%%%%%%%%%%%%%%%%%%%%%%%%%%%%%%%%%%%%%%%%%%%%%%%%%%%%%%%%%%%%%%%%%%%%%%%%%%%%%%%%%%%%%%%%%%%%%%%%%%%%%%%%%%%%%%%%%%%%
\section{Modification des membres du corps humain}
%\subsection{Récupération de l'image du corps humain}
%SDK de la kinect avec skelette
%\subsection{Préparation de l'image}
%suppression du bruit et diminution du nombre de point -> pcl
%\subsection{Segmentation du corps humain}
%distance geodesic avec dikjstra -> pb nombre de voisin
%superpixel
%\subsection{Reconnaissance des parties du corps}
%descripteur D2 avec model 3D
%FPFH
%\subsection{Appariement d'un model 3D}
%scale à partir d'une bounding box
%ICP
%moment d'inertie
%\subsection{Résultat des expérimentations}
%montrer plusieurs images de la segmentation et montrer ce qui ne convient pas
%dire que les méthode utilisé seront utilisé dans la suite du projet 
\subsection{Objectif}
%on cherche a ne pas utiliser la kinect
Cette première application a plusieurs objectifs. Dans un premier temps, nous souhaitons réaliser 
une segmentation sans utiliser les outils fournis pas la Kinect dans le but d'obtenir une méthode plus
précise ou tout du moins permettant d'améliorer une partie du procédé de segmentation. Comme nous l'avons vu dans 
l'état de l'art, les méthodes utilisées par la Kinect ont évolué et sont devenu plus stable et plus précise.
Le second objectif et de remplacer un nuage de point représentant un membre par un modèle 3D de ce même membre
mais ayant une forme différente. Cette première partie du stage me permet surtout d'aborder des concepts qui me 
serviront dans la second application qui est le but premier de ce stage.

\subsection{Délimitation du corps humain et minimisation de la quantité de donnée}
%prétraitement
%utilisation voxel pour réduire le nombre de point
%utilisation d'outils pour supprimer les outlier
%utilisation de threshold
%suppression d'objet à la main
%utilisateur peut les modifier lui meme
%finalement utilisation de la Kinect pour la delimitation
Durant cette première phase nous travaillons sur un nuage de point et non sur les informations de l'image de profondeur.
Comme nous l'avons dit précédemment, l'ensemble des informations n'est pas pértinente pour les traitements que nous souhaitons
réaliser. Dans cette première application il nous faut dabord supprimer l'ensemble des informations qui ne se rapporte pas 
au corps humain. La première solution que j'ai utilisé et qui était la plus simple a été de mettre deux seuil afin de 
supprimer les informations qui sont au-dessus d'un certain seuil ainsi que les informations qui sont en-dessous d'un second
seuil. Ces seuils sont les distances, en mètre, dans lequel l'utilisateur doit se trouver. Cette méthode est utilisé dans 
l'application ReconstructMe\footnote{http://reconstructme.net} qui est une application permettant de construire un modèle
3D complet à partir de données fournit par plusieurs images provenant d'une caméra 3D. Comme pour cette application, j'ai
décidé de laisser la possibilité à l'utilisateur de changer les seuils en fonction de sont besoin.\\

%TODO mettre une image délimitant le corps humain avec des seuils
\begin{figure}[!ht]
  \begin{center}
    \includegraphics[width=5cm]{image/wait.png}
    \caption{Résultat d'une segmentation par seuillage}
    \label{fig:seuillage}
  \end{center}
\end{figure}

Nous pouvons voir sur la Fig. \ref{fig:seuillage} que le seuil permet effectivement de supprimer beaucoup d'information
correspondant à l'environnement, mais qu'il reste beaucoup de bruit du à la qualité de l'acquisition de la caméra. La librairie
PCL\cite{PCL} nous fournit beaucoup d'outil pour ce genre de problèmatique. Il y a une classe appelé StatisticalOutlierRemoval
qui permet de supprimer les points supposé être du bruit. Pour cela cette classe calcul la distance moyenne d'un point avec son
voisinnage et si cette moyenne est trop élevé, elle supprime le point en question.\\

%TODO mettre une image délimitant le corps humain avec des seuils et avec la suppression du bruit
\begin{figure}[!ht]
  \begin{center}
    \includegraphics[width=5cm]{image/wait.png}
    \caption{Résultat d'une segmentation par seuillage avec suppression du bruit}
    \label{fig:seuillageOutlier}
  \end{center}
\end{figure}

Pour la suite des traitements que nous souhaitons réalisé, il est préférable d'avoir un minimum d'information et de ne garder
que ce qui est pertinant. Le corps humain que nous avons réussi à délimité comporte encore beaucoup trop de données. Le nombre
de point fournit par la Kinect est très important et très concentré, il est possible de supprimer des points qui sont trop 
proche les uns les autres. La encore PCL\cite{PCL} peut nous aider avec la classe VoxelGrid. Cette classe crée une grille de
voxel sur le nuage de point dont la taille est définit par l'utilisateur. L'ensemble des points à l'intérieur d'un voxel sont 
approximé en un point qui correspond au centroïde du voxel. Grâce à l'ensemble de ces traitements nous ne gardons que l'information 
essentiel à nos traitement.

\subsection{Calcule de la distance géodésique}
Ma première idée pour segmenter le corps humain est d'utiliser la distance géodesique. Y. Liu et al\cite{GIF} montre que la distance
géodesique pour une même personne quelque soit sa posture est toujours la même pour les points de ces membres. Le seul problème 
est que plusieurs membres ont la même distance géodésique, on ne peut donc pas associer un membre à une distance directement.
Cependant, s'il est possible de seuiller le corps et d'en calculer des descripteurs, nous pourrons déterminer quel nuage de point
correspond à quel membre.\\

Avant de calculer la distance géodésique du corps humain, nous avons besoin de créer un maillage sur le nuage de point. Pour cela
j'ai repris le principe utilisé pour le descripteur FPFH\cite{FPFH} pour la sélection des voisins. Dans un premier temps je calcule
la distance euclidienne de chacun des points du nuage de point avec tous les autres. Pour la création du voisinage je considère non
seulement le nombre maximum de point à prendre en compte, mais aussi la distance maximum d'un point avec les autres. Nous avons eu
plusieurs problème lorsque nous n'avions pas mis la second condition, car lorsque la main de l'utilisateur était trop proche de sa 
jambe certain point de la main avait des voisins dans la jambe.\\

%TODO voir si on met des exemples de maillage defectueux 

A cette étape de l'algorithme nous avons donc un maillage et les distances euclidiennes de chaque point avec son voisinnage. Pour calculer
la distance géodésique du centroïde du nage de point avec chaque point nous allons utilisé l'algorithme de Dijkstra\cite{dijkstra}.
Il faut donc trouver le plus court chemin du centroïde jusqu'au point dont on veut calculer la distance géodésique en passant par le
maillage que nous avons construit précédemment. A chaque fois qu'un point est ajouté au chemin emprunté par l'algorithme il faut 
additionner la distance euclidienne entre ce point et le précédent pour obtenir un résultat final correspondant à la distance 
géodésique.\\

%TODO ajout d'une image d'un corps humain avec un chemin ?
\begin{figure}[!ht]
  \begin{center}
    \includegraphics[width=5cm]{image/wait.png}
    \caption{Exemplre de chemin parcouru par l'algorithme de calcul de la distance géodésique pour un point}
    \label{fig:cheminGeodesique}
  \end{center}
\end{figure}

%TODO refaire les images
\begin{figure}[!ht]
  \begin{center}
    \includegraphics[width=6.5cm]{image/geodesic1.PNG}
    \includegraphics[width=6cm]{image/geodesic2.PNG}
    \caption{Résultat du calcul de la distance géodésique sur l'ensemble du nuage de point. Le point vert correspond au centroïde du
    nuage de point. Plus la couleur des points est proche de rouge, plus les points sont proches du centroïde.}
    \label{fig:geodesique}
  \end{center}
\end{figure}

Sur les images de la Fig. \ref{fig:geodesique} on peut voir que la distance géodésique n'est pas aussi précise qu'on le souhaiterait.
Dans la première image on voit que la distance géodésique n'est pas la même sur le bras gauche et sur le bras droit. Cette imprécision
vient de la qualité du maillage, certains points ne passent pas par le corps et traverse dans le vide du ventre au bras, ce qui réduit
la distance géodésique au niveau des mains. Dans la second image, on voit que la position du centroïde est très importante, la position
du centroïde change en fonction de ce que l'on voit du corps humain, ce qui modifie également la distance géodésique. Un simple seuillage
n'est donc pas suffisant pour segmenter le corps humain.

\subsection{Segmentation du corps humain}
%recherche du point le plus éloigné puis suppression d'un groupe de point ....
%voir si on parle des superpixels
Malgrès un léger manque de précision de la par du calcul de la distance géodésique sur le corps humain, nous avons testé une
méthode de segmentation qui ne prend pas en compte cette imprécision. Le principe de cette segmentation est de détecter le
membre le plus éloigné puis de le supprimer dans la recherche des autres membres. Pour cela, notre algorithme recherche le point
dont la distance géodésique est la plus grande et forme une zone autour de ce point. Cette zone a un rayon prédéfini qui sera le 
même pour chaque membre. Elle sera enregistré en mémoire et les points de cette zone seront supprimé du nuage de point du corps humain dans
lequel nous effectuons notre recherche.\\

\begin{figure}[!ht]
  \begin{center}
    \includegraphics[width=3cm]{image/humanFootR.png}
    \includegraphics[width=3cm]{image/humanFootL.png}
    \includegraphics[width=3cm]{image/humanHandR.png}
    \includegraphics[width=3cm]{image/humanHandL.png}
    \caption{Premières étape de l'algorithme de segmentation du corps humain. Le point vert correspond au point le plus éloigné et la zone rouge
    l'ensemble des points concidérés comme fesant partie du membre.}
    \label{fig:geodesique}
  \end{center}
\end{figure}

Cette méthode est efficace pour reconnaitre le bout des membres et la tête, mais le reste des parties du corps est moins précis, notamment au niveau
des épaules où le point de la zone est très instable et peut se retrouver au niveau du torse. De plus la taille des zones dépend de la 
physionomie de la personne devant la Kinect. 
%Nous décidons donc de nous concentrer sur les parties que nous arrivons à détecter pour la 
%suite de nos traitements, c'est-à-dire les mains et la tête.

%\subsection{Calcule de descripteurs}
%d2
\subsection{SDK de la Kinect}
%récuperation du corps de la personne depuis une image kinect -> a partir des info de la kinect -> le mettre plutot dans les travaux réalisé
%labelisation en fonction de la distance du point avec le joint
Etant donné que nos résultats pour la segmentation du corps ne sont pas suffisemment précis pour la suite de nos traitements et par faute de
temps, nous avons décidé d'utiliser les outils fournis avec la Kinect pour continuer le projet. Grâce à la caméra de Microsoft, nous pouvons
récupérer le squelette de l'utilisateur dont les articulations sont labelisé avec le nom de la partie du corps humain à laquel elle appartient
(voir Fig. \ref{fig:kinect}.a). De plus, la Kinect nous permet de séparer les points qui appartiennent à l'utilisateur de ceux qui appartiennent
à l'environnement (voir Fig. \ref{fig:kinect}.b).\\

\begin{figure}[!ht]
  \begin{center}
    \includegraphics[width=2cm]{image/kinectSkeleton.png} 
    \includegraphics[width=8cm]{image/seg1.PNG}
    \caption{a) Squelette fournit par la caméra Kinect et b) séparation du corps humain de l'environnement}
    \label{fig:kinect}
  \end{center}
\end{figure}

Grâce à ce squelette et au nuage de point nous pouvons segmenter le corps humain. Pour cela, pour chaque articulation nous définissons une zone
dont la taille est variable en fonction de l'articulation. Les points du nuage sont labelisé en fonction de leur distance avec les articulations.
Donc un point prend le label de l'articulation dont il est le plus proche.

\begin{figure}[!ht]
  \begin{center}
    \includegraphics[height=4cm]{image/lab1.PNG} 
    \includegraphics[height=4cm]{image/lab2.PNG}
    \includegraphics[height=4cm]{image/lab3.PNG}
    \caption{Résultat obtenu avec la segmentation du corps humain via les outils de la Kinect}
  \end{center}
\end{figure}

Nous pouvons voir que comme nous l'attendions, le résultat de cette segmentation est bien meilleur puisque nous avons à notre disposition
beaucoup plus d'information qu'auparavant grâce au squelette de l'utilisateur. Même si cette segmentation n'est pas parfaite comme on peut
le voir dans la première image, elle reste suffisante pour nos futures traitement. En plus de la segmentation nous connaissons déjà le nom
des membres grâce aux labels des articulations, nous n'avons donc pas besoin de passer par une étape de reconnaissance des parties du
corps, car celle-ci a déjà été réalisé par les outils de la Kinect. Il ne reste plus qu'à supprimer le bruit présent dans les membres, les points
mal labeliser, grâce à la classe StatisticalOutlierRemoval que nous avons vu précédemment.

\subsection{Positionnement}
%icp, fpfh
%dire qu'il n'est pas possible d'utiliser la distance geodesic pour le positionnement a cause du temps et qu'on ne peut calculer la distance geodesic
%sur le mesh de remplacement


%%%%%%%%%%%%%%%%%%%%%%%%%%%%%%%%%%%%%%%%%%%%%%%%%%%%%%%%%%%%%%%%%%%%%%%%%%%%%%%%%%%%%%%%%%%%%%%%%%%%%%%%%%%%%%%%%%%%%%%%%%%
\section{Reconstruction d'un environnement intérieur}
%\subsection{Segmentation de l'environnement intérieur}
%segmentation par l'utilisateur
%\subsection{Création d'une base de connaissance}
%utilisation opencv
%creation du dictionnaire
%calcul du descripteur des objets -> detail sur le nombre d'objet et sur le nombre d'exemplaire
%test sur une base composé de mesh et sur une autre composé de nuage de point
%\subsection{Reconnaissance des objets} 
%simple comparaison avec le descripteur
%TODO voir ce qu'on peut mettre dans cette section

\subsection{Objectif}
%interface
Le but de cette seconde application est de pouvoir modifier un environnement intérieur au travers d'une interface
simple et intuitive. 
Pour cela, nous allons dans un premier temps récupérer un nuage de points d'une pièce intérieure. Puis à partir
de ce nuage de points, nous allons segmenter la pièce afin de pouvoir détecter des objets à l'intérieur de celle-ci.
L'utilisateur doit ensuite sélectionner un objet, ce qui déclenchera l'apparition d'une liste contenant un ensemble de modèle 3D
d'objets équivalents à celui sélectionné. L'utilisateur n'a plus qu'à choisir l'objet qui lui convient dans la liste,
afin de l'ajouter dans un autre environnement 3D présent dans l'application.\\

Lors de ce scénario, nous pouvons voir qu'il y a deux grandes étapes qu'il faudra réaliser dans l'application. Tout d'abord, il faut 
segmenter le nuage de points de la pièce, dans le but de détecter des objets. Puis il faut reconnaître les objets détectés afin
d'afficher la bonne liste d'objets 3D. Nous avons décidé, pour faciliter le développement, de laisser l'utilisateur sélectionner
l'objet qu'il souhaite dans le nuage de points, afin de ne pas avoir à développer la détection d'objet dans le nuage de points.
Pour cela, nous nous inspirons des travaux de T. Shao et al\cite{interactiveSeg} et de J. Xiao et al\cite{interactionSeg2}.
Il nous reste donc à reconnaitre l'objet sélectionné par l'utilisateur.

\subsection{Création de la base d'apprentissage}
Afin de pouvoir reconnaître un objet, nous avons besoin d'une base d'apprentissage contenant les descripteurs d'un ensemble
d'objets. Pour créer cette base, nous utilisons la représentation du \og bag of word \fg ainsi que la méthode d'apprentissage 
automatique SVM\cite{SVM} disponible dans la librairie opencv. Pour la création de notre base, nous utilisons une vingtaine
d'images de six objets différents. Les nuages de points dont nous nous sommes servis pour la création de notre base, viennent de 
K. Lai et al\cite{Base1} issue de M. Firman\cite{generalBase} qui a regroupé un ensemble de bases d'objets provenant de caméra 3D.
La base que nous utilisons est composée de plusieurs exemplaires de chaque objet, et chaque exemplaire d'objet est représenté par 
plusieurs images représentant plusieurs angles de vue de l'objet. L'acquisition des objets a été réalisée avec une caméra 
Kinect v1.\\

Le descripteur que nous utilisons pour cette application est le descripteur FPFH\cite{FPFH} qui est l'un des plus efficaces dans la
reconnaissance d'objet à partir de nuage de points. Nous calculons ce descripteur sur l'ensemble des images de notre base afin de
créer le dictionnaire de notre \textit{bag of words}. A cette étape du développement, nous avons créé le dictionnaire, il faut
alors recalculer le descripteur sur chacun des objets afin de déterminer les caractèristiques de chaqu'un d'entre eux. Cela nous
permet d'obtenir un histogramme pour chaque objet comportant les caractèristiques qu'il contient et le nombre de fois où il
apparaît. Il ne reste plus qu'à utiliser le descripteur du \textit{bag of word} dans SVM afin de finaliser la création de 
notre base d'apprentissage. Celle-ci est donc un vecteur d'une taille égale au nombre de classes, donc six, 
composé d'histogramme dont la taille est égale au nombre de caractèristiques.\\

\begin{figure}[!ht]
  \begin{center}
    \includegraphics[height=8cm]{image/schemaBase.png}
    \caption{Schéma de la création de la base d'apprentissage}
  \end{center}
\end{figure}

Nous créons une seconde base d'apprentissage utilisant le même principe que précédemment, mais qui utilise des images que nous 
produisons nous même avec notre caméra Kinect v2. Cette base comporte quatre objets, chacun représenté par dix à quinze angles de vue différent.
L'intérêt de cette seconde base est de vérifier que la qualité de l'acquisition ne fait pas varier significativement les résultats de
la reconnaissance d'objet. Cependant, le fait de créer nous même notre base ne nous permet pas d'avoir autant de donné que dans la base de 
K. Lai et al\cite{Base1} à cause du temps que cela prend de créer une tel base.

\subsection{Interface utilisateur}
L'interface utilisateur comprend deux fenêtres. La première comporte une vue avec le nuage de points récupéré à partir de la Kinect
et un ensemble d'options. Parmi ces options, nous avons la navigation dans le nuage de points et le choix de la technique de 
sélection. La première sélection est une simple sélection rectangulaire où l'ensemble des données à l'intérieur d'un rectangle formé
par deux points est sélectionné. La seconde est une sorte de pinceau sélectionnant les données en dessous de la souris, ainsi que celles dans 
un voisinage prédéfini autour de celle-ci. Lorsque l'utilisateur a sélectionné un objet et que l'application l'a reconnu, un nombre \textit{n} de cases 
avec des objets apparaît. Les objets dans ces cases sont les mêmes que l'objet sélectionné. Lorsque l'utilisateur clique sur l'une 
de ces cases, l'objets correspondant est ajouté dans la seconde fenêtre.
La seconde fenêtre contient l'environnement final dans lequel il y a déjà une piéce modèlisée avec des objets 3D. L'objet sélectionné est ajouté dans cette environnement. Il est ensuite possible de déplacer l'objet dans l'environement virtuel en
cliquant dessus afin de le positionner à l'endoit voulu dans la pièce.

\begin{figure}[!h]
  \begin{center}
    \includegraphics[height=7cm]{image/appliObjet.PNG}
    \caption{Interface de la seconde application}
  \end{center}
\end{figure}

\begin{figure}[!h]
  \begin{center}
    \includegraphics[height=4.5cm]{image/selection1.png}
    \includegraphics[height=4.5cm]{image/selection2.png}
    \caption{Schéma des deux type de sélection disponible dans l'apllication. Les points verts
    sont les points sélectionnés et en noire les autres.}
  \end{center}
\end{figure}

\subsection{Reconnaissance d'objet}
%TODO parler des résultats de la reconnaissance
Une fois la base d'apprentissage et l'interface créée, nous ajoutons le code de la reconnaissance de forme dans l'application.
Pour cela, nous filtrons les données que l'utilisateur a séléctionné. Nos méthodes de sélection n'étant pas très précises, il arrive
couramment que l'utilisateur sélectionne du bruit ou des parties de l'environnement. Pour filtrer les données, nous utilisons
les deux mêmes classes que dans l'application précédente, c'est-à-dire \textit{StatisticalOutlierRemoval} et \textit{VoxelGrid}.
Lorsque le filtrage est réalisé, il nous faut calculer le même descripteur que dans notre base sur les données sélectionnées par 
l'utilisateur, puis envoyer le résultat dans notre SVM afin qu'il détermine la classe de l'objet.\\

En utilisant les objets de la base de K. Lai et al\cite{Base1} nous obtenons des résultats très insuffisant de la reconnaissance 
des objets provenant de notre caméra Kinect v2. Les différences entre nos données et celle provenant de cette base viennent
de notre sélection et de la caméra. Dans les données de la base, le bruit est presque null tandis que dans notre sélection il 
y en a un peu plus, malgrès que nous en supprimions un maximum grâce à la librairie PCL. La principal différence entre la caméra
de K. Lai et al\cite{Base1} et la notre est la résolution de la caméra qui est plus grande dans notre cas. Avec ces paramètres, ne parvient
à reconnaitre qu'un objet sur les six testés et cela dépend du nombre de mot que nous mettons dans le dictionnaire. Cette première base nous 
à permis de déterminer que le nombre de mot idéal qu'il faut mettre dans le dictionnaire pour notre application est de deux milles.\\

\subsection{Travaux futurs et applications}
L'application finale de ce projet est assez complète et très fonctionnelle, cependant il existe plusieurs extensions qu'il 
est possible d'ajouter au projet. Par exemple, notre liste d'objets proposés à l'utilisateur pour ajouter à la 
scène 3D n'est pas assez conséquente. Nous pourrions ajouter plus d'objets en récupérant une base venant de grands magasins
de meubles par exemple. Ainsi l'utilisateur pourrait potentiellement retrouver le meuble qu'il a scanné. Cela pourrait
déboucher sur une application beaucoup plus importante, où l'environnement dans lequel nous souhaitons ajouter des meubles
représente la future maison d'une personne. Si cette personne souhaite voir comment elle pourrait agencer ses meubles dans 
son prochain investissement, elle n'aurait qu'à récupérer un modèle 3D des lieux et scanner l'ensemble de ses meubles pour
les ajouter dans la scène.


%%%%%%%%%%%%%%%%%%%%%%%%%%%%%%%%%%%%%%%%%%%%%%%%%%%%%%%%%%%%%%%%%%%%%%%%%%%%%%%%%%%%%%%%%%%%%%%%%%%%%%%%%%%%%%%%%%%%%%%%%%%
%\section{Evaluation des méthodes}
%TODO aucune méthode d'évaluation déterminé

%%%%%%%%%%%%%%%%%%%%%%%%%%%%%%%%%%%%%%%%%%%%%%%%%%%%%%%%%%%%%%%%%%%%%%%%%%%%%%%%%%%%%%%%%%%%%%%%%%%%%%%%%%%%%%%%%%%%%%%%%%%
\section{Conclusion}
Durant ce stage, j'ai effectué deux projets permettant de réaliser la segmentation d'objets complexes comme le corps
humain ainsi qu'une scène intérieure à partir d'image 3D provenant d'une caméra Kinect. L'objectif était de créer
des interfaces permettant de faciliter la tâche des professions artistiques en leur permettant d'utiliser 
des objets du monde réel dans un environement virtuel. Dans la première application sur le corps humain,
nous avons cherché à récupérer les informations du corps de la personne en face de la caméra et segmenter
ces informations. Les informations ainsi segmentées pouvaient ensuite être remplacées par l'utilisateur afin de 
pouvoir mettre des éléments virtuels sur un corps venant du monde réel. Pour réaliser ce projet, nous avons 
utilisé les fonctionnalités fournis dans le SDK de la Kinect pour filtrer les données. Au vu de la différence
des données entre le nuage de points que nous récupérons avec notre caméra et les modèles 3D que nous utilisons,
nous avons privilégié une méthode basée sur l'utilisation des caractèristiques de la boîte englobante pour
déterminer la position et l'orientation du nuage de points de la partie du corps à remplacer. Ces informations
nous permettent ainsi de transformer le modèle 3D pour que celui-ci soit exactement à la même position que
le premier nuage de points. Les limites de notre méthode sont qu'une boîte englobante est une représentation 
beaucoup trop générale pour une forme, et donc l'orientation du nouveau modèle possède souvent une erreur de 90\degre
sur l'un des trois axes de la boîte. Pour l'application finale, nous avons pensé proposer à l'utilisateur de 
corriger ce problème en sélectionnant l'axe qui fait défaut ainsi l'application se chargerait d'appliquer la
bonne rotation au modèle.\\

La seconde applications avait pour but de prendre un environnement virtuel dans lequel nous pouvons rajouter des éléments 
du monde réel. Ces objets du monde réel viennent de l'acquisition d'une pièce via la caméra Kinect. Là encore, le processus nécessitait 
la segmentation des données afin de détecter les objets présents dans la pièce réelle. La segmentation est réalisée 
par l'utilisateur qui doit sélectionner l'objet qu'il veut importer dans le monde virtuel. Pour que l'application
reconnaisse l'objet qui a été sélectionné, nous avons utilisé un algorithme d'apprentissage automatique, les SVM,
combiné avec l'approche du \textit{bag of words} formé à partir du descripteur FPFH.\\
%TODO ajout des résultat de la méthode.

Lors de ce stage, j'ai eu l'occasion de perfectionner les connaissances que j'avais acquises lors de mon master.
Les notions que j'avais vues de reconnaissance d'objet m'ont évidemment était très utile durant le stage, mais 
beaucoup de concepts n'ont pas été approfondis durant les cours par manque de temps. J'ai pu perfectionner mes 
connaissances concernant les algorithmes d'apprentissage automatique et j'ai pu voir le fonctionnement de plusieurs
descripteurs.\\ 

Le plus compliqué lors de ce stage a été de comprendre et tester des méthodes proposées dans des publications scientifiques.
Certaines méthodes n'étaient pas très complexes et faciles à tester, mais de nombreuses publications se reposaient sur des connaissances
mathématiques qui n'ont pas été vues durant mon cursus scolaire. Cependant, cette expérience m'a appris à réaliser un état de l'art des 
solutions existantes sur un sujet très spécifique et à tester les méthodes proposées dans la littérature. Ma seconde difficulté,
lors de ce stage, est que le sujet était assez vaste et qu'il ne fallait pas se perdre dans des méthodes qui s'éloignaient du sujet.
Cela m'a appris à rester concentré sur l'objectif principal en prévoyant et en organisant les étapes du projet.\\ 

Ce stage m'a fait redécouvrir l'environnement de la recherche dans lequel je souhaitais m'orienter. Bien que j'ai réussi à surmonter
les difficultés que j'ai rencontrées durant ce stage, il m'a permis de constater que l'environnement ne me convenait pas. La partie 
\og état de l'art \fg, bien que très intéressante, a été assez complexe à réaliser. De plus, j'ai ressenti une certaine frustration
concernant les résultats du projet. J'ai passé énormément de temps à tester des méthodes difficiles à implémenter et je n'obtenais
que rarement des résultats intéressants pour mon projet. Ce stage m'a permis de comprendre que le monde de l'entreprise me
convenait mieux, grâce à son organisation et à sa structure qui permet de fournir des résultats 
rapides. Je pense qu'à l'heure actuelle, je ne suis pas encore capable de structurer un projet pour
le mener à bien avec autant d'autonomie. Même si la recherche est un travail d'équipe, les projets 
s'organisent souvent seul, et c'est sur ce point qu'il me reste encore à progresser.

%resultat des deux projets
%ce que j'ai utilisé
%ce que j'ai appris
%comment me servir de mon stage pour après mes études

\newpage
\bibliographystyle{ieeetr} % or try abbrvnat or unsrtnat or plainnat
\bibliography{biblio}

\end{document}
